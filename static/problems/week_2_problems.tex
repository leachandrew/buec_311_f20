\documentclass[11pt,]{article}
\usepackage[margin=1in]{geometry}
\newcommand*{\authorfont}{\fontfamily{phv}\selectfont}
\usepackage[]{mathpazo}

\usepackage{xcolor}

%TIKZ and Flow chart material
\usepackage{tikz}
\usetikzlibrary{shapes.geometric, arrows}
\tikzstyle{startstop} = [rectangle, rounded corners, minimum width=3cm, minimum height=1cm,text centered, draw=black, fill=red!30]
\tikzstyle{io} = [trapezium, trapezium left angle=70, trapezium right angle=110, minimum width=3cm, minimum height=1cm, text centered, draw=black,fill=white]
\tikzstyle{process} = [rectangle, minimum width=3cm, minimum height=1cm, text centered, draw=black, fill=white]
\tikzstyle{decision} = [diamond, minimum width=3cm, minimum height=1cm, text centered, draw=black, fill=white]
\tikzstyle{smalldecision} = [diamond, minimum width=1cm, minimum height=1cm, text centered, draw=black, fill=white]
\tikzstyle{arrow} = [thick,->,>=stealth]
\tikzstyle{line} = [thick,-]
\tikzstyle{dot} = [circle,inner sep=0.5pt,draw=black, fill=black]


%\tikzstyle{startstop} = [rectangle, rounded corners, minimum width=3cm, minimum height=1cm,text centered, %draw=black, fill=red!30]
%\tikzstyle{io} = [trapezium, trapezium left angle=70, trapezium right angle=110, minimum width=3cm, %minimum height=1cm, text centered, draw=black,fill=white]
%\tikzstyle{process} = [rectangle, minimum width=3cm, minimum height=1cm, text centered, draw=black, %fill=white]
%\tikzstyle{decision} = [diamond, minimum width=3cm, minimum height=1cm, text centered, draw=black, %fill=green!30]
%\tikzstyle{arrow} = [thick,->,>=stealth]




\usepackage{abstract}
\renewcommand{\abstractname}{}    % clear the title
\renewcommand{\absnamepos}{empty} % originally center
\newcommand{\blankline}{\quad\pagebreak[2]}

\providecommand{\tightlist}{%
  \setlength{\itemsep}{0pt}\setlength{\parskip}{0pt}}
\usepackage{longtable,booktabs}

\usepackage{parskip}
\usepackage{titlesec}
\titlespacing\section{0pt}{12pt plus 4pt minus 2pt}{6pt plus 2pt minus 2pt}
\titlespacing\subsection{0pt}{12pt plus 4pt minus 2pt}{6pt plus 2pt minus 2pt}

\titleformat*{\subsubsection}{\normalsize\itshape}

\usepackage{titling}
\setlength{\droptitle}{-.25cm}

%\setlength{\parindent}{0pt}
%\setlength{\parskip}{6pt plus 2pt minus 1pt}
%\setlength{\emergencystretch}{3em}  % prevent overfull lines

\usepackage[T1]{fontenc}
\usepackage[utf8]{inputenc}

\usepackage{fancyhdr}
\pagestyle{fancy}
\usepackage{lastpage}
\renewcommand{\headrulewidth}{0.3pt}
\renewcommand{\footrulewidth}{0.0pt}
%\lhead{\footnotesize Problem Set \#2 }
\lhead{\footnotesize BUEC 311: Problem Set \#2, Mechanics of Supply and Demand}
\rhead{\footnotesize \today}
\lfoot{\small \copyright }
\cfoot{}
\rfoot{\small \thepage/\pageref*{LastPage}}

\fancypagestyle{firststyle}
{
\renewcommand{\headrulewidth}{0pt}%
   \fancyhf{}
   \fancyfoot[C]{\thepage/\pageref*{LastPage}}
}

%\def\labelitemi{--}
%\usepackage{enumitem}
%\setitemize[0]{leftmargin=25pt}
%\setenumerate[0]{leftmargin=25pt}




\makeatletter
\@ifpackageloaded{hyperref}{}{%
\ifxetex
  \usepackage[setpagesize=false, % page size defined by xetex
              unicode=false, % unicode breaks when used with xetex
              xetex]{hyperref}
\else
  \usepackage[unicode=true]{hyperref}
\fi
}
\@ifpackageloaded{color}{
    \PassOptionsToPackage{usenames,dvipsnames}{color}
}{%
    \usepackage[usenames,dvipsnames]{color}
}
\makeatother
\hypersetup{breaklinks=true,
            bookmarks=true,
            pdfauthor={},
             pdfkeywords = {},
            pdftitle={Problem Set \#2: Mechanics of Supply and Demand},
            colorlinks=true,
            citecolor=blue,
            urlcolor=blue,
            linkcolor=magenta,
            pdfborder={0 0 0}}
\urlstyle{same}  % don't use monospace font for urls


\setcounter{secnumdepth}{0}


%



\usepackage{setspace}

\title{\vspace{-1.5cm}\Large{BUEC 311: Business Economics, Organization and Management}\medskip\\\Large{Problem Set \#2}
\medskip\\\Large{Mechanics of Supply and Demand}
}
\date{\vspace{-.75cm}\Large{\today}}

\definecolor{light-gray}{gray}{0.8}


\begin{document}

\vspace{-5cm}\maketitle
 \tikz [remember picture,overlay]
    %\node[yshift=-1.65cm,xshift=0cm] at (current page.north)
    %\node[yshift=-1.65cm,xshift=0cm] at (current page.north)
        %or: (current page.center)
        \node[yshift=-1cm,xshift=6.5cm] at (current page.north west)
        %{\includegraphics[width=3in]{UA-ASB-COLOUR.png}};
        {\includegraphics[width=.5\paperwidth]{../images/UA-ASB-COLOUR.png}};
\vspace{-.75cm}		
		\thispagestyle{firststyle}



This week's problems are meant to give you some practice on the basic
tricks you'll need to manipulate supply and demand curves and calculate
equilibrium conditions. In this questions, I've given you multiple steps
intentionally to build habits like checks and double-checks that you'll
want to use. I won't always give these to you on exam questions, but
best to build the habit now. The questions here are intended to be more
difficult as you progress through the problem set.

\begin{enumerate}
\def\labelenumi{\arabic{enumi}.}
\item
  Suppose that my personal demand curve for coffee is given by
  \(Q=4-\frac{2p}{3}\).

  \begin{enumerate}
  \def\labelenumii{\alph{enumii})}
  \item
    Solve for and draw a graph of the inverse demand curve for this
    function (i.e p as a function of Q).
  \item
    Check your work. Use the demand curve to figure out the price at
    Q=2, and then show, both algebraically and graphically, that the
    same result is obtain from your inverse demand function.
  \item
    How many cups of coffee would you expect me to consume per day at a
    price of \$2 per cup? Assume that I can't buy fractions of a cup.
  \item
    If I arrive in a meeting and there's free coffee, but I've already
    had 3 cups of coffee today, should I have another cup? Why or why
    not?
  \end{enumerate}
\item
  Now, suppose there are 100 people with the same coffee demand as I
  have.

  \begin{enumerate}
  \def\labelenumii{\alph{enumii})}
  \tightlist
  \item
    What's the market demand curve for coffee?
  \item
    Check your work: show me that your calculated market demand predicts
    100 times the amount that I would consume at price p=3.
  \end{enumerate}
\item
  Now, let's imagine you've got a market in the School of Business of
  1000 people who each have the same preferences for coffee that I have.
  Let's also assume that the marginal cost of supplying coffee to the
  market is \(Q=\frac{2000\,p}{3}\).

  \begin{enumerate}
  \def\labelenumii{\alph{enumii})}
  \tightlist
  \item
    Derive the market demand curve
  \item
    Check that your market demand is equal to 1000 times individual
    demand at price p=2.
  \item
    Find the equilibrium price and quantity of coffee in the market.
  \item
    Show that individual quantity demanded at the equilibrium price is
    1/1000th of the total market quantity demanded.
  \item
    Show that the quantity supplied at the equilibrium price is the
    equilibrium quantity.
  \end{enumerate}
\end{enumerate}

\newpage

\begin{enumerate}
\def\labelenumi{\arabic{enumi}.}
\setcounter{enumi}{3}
\tightlist
\item
  In Episode 3 of the Think Like an Economist podcast, The Marginal
  Principle, from 5:48 to 9:45, we hear a discussion of hiring decisions
  framed around the marginal benefits and marginal costs of hiring
  workers in a coffee shop. The estimate of the marginal benefits of
  hiring a third barista includes an estimated benefit of \$100, but
  also includes some other considerations.
\end{enumerate}

In your own words, describe why this dollar estimate doesn't accurately
reflect the owner's assessment of the benefits of hiring of third worker
as revealed through his own decisions, and how the assessment of
marginal benefits might have been done from the outset.




\end{document}

