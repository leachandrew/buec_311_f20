\documentclass[11pt,]{article}
\usepackage[margin=1in]{geometry}
\newcommand*{\authorfont}{\fontfamily{phv}\selectfont}
\usepackage[]{mathpazo}

\usepackage{amssymb,amsmath}
\usepackage{xcolor}

%TIKZ and Flow chart material
\usepackage{tikz}
\usetikzlibrary{shapes.geometric, arrows}
\tikzstyle{startstop} = [rectangle, rounded corners, minimum width=3cm, minimum height=1cm,text centered, draw=black, fill=red!30]
\tikzstyle{io} = [trapezium, trapezium left angle=70, trapezium right angle=110, minimum width=3cm, minimum height=1cm, text centered, draw=black,fill=white]
\tikzstyle{process} = [rectangle, minimum width=3cm, minimum height=1cm, text centered, draw=black, fill=white]
\tikzstyle{decision} = [diamond, minimum width=3cm, minimum height=1cm, text centered, draw=black, fill=white]
\tikzstyle{smalldecision} = [diamond, minimum width=1cm, minimum height=1cm, text centered, draw=black, fill=white]
\tikzstyle{arrow} = [thick,->,>=stealth]
\tikzstyle{line} = [thick,-]
\tikzstyle{dot} = [circle,inner sep=0.5pt,draw=black, fill=black]


%\tikzstyle{startstop} = [rectangle, rounded corners, minimum width=3cm, minimum height=1cm,text centered, %draw=black, fill=red!30]
%\tikzstyle{io} = [trapezium, trapezium left angle=70, trapezium right angle=110, minimum width=3cm, %minimum height=1cm, text centered, draw=black,fill=white]
%\tikzstyle{process} = [rectangle, minimum width=3cm, minimum height=1cm, text centered, draw=black, %fill=white]
%\tikzstyle{decision} = [diamond, minimum width=3cm, minimum height=1cm, text centered, draw=black, %fill=green!30]
%\tikzstyle{arrow} = [thick,->,>=stealth]




\usepackage{abstract}
\renewcommand{\abstractname}{}    % clear the title
\renewcommand{\absnamepos}{empty} % originally center
\newcommand{\blankline}{\quad\pagebreak[2]}

\providecommand{\tightlist}{%
  \setlength{\itemsep}{0pt}\setlength{\parskip}{0pt}}
\usepackage{longtable,booktabs}

\usepackage{parskip}
\usepackage{titlesec}
\titlespacing\section{0pt}{12pt plus 4pt minus 2pt}{6pt plus 2pt minus 2pt}
\titlespacing\subsection{0pt}{12pt plus 4pt minus 2pt}{6pt plus 2pt minus 2pt}

\titleformat*{\subsubsection}{\normalsize\itshape}

\usepackage{titling}
\setlength{\droptitle}{-.25cm}

%\setlength{\parindent}{0pt}
%\setlength{\parskip}{6pt plus 2pt minus 1pt}
%\setlength{\emergencystretch}{3em}  % prevent overfull lines

\usepackage[T1]{fontenc}
\usepackage[utf8]{inputenc}

\usepackage{fancyhdr}
\pagestyle{fancy}
\usepackage{lastpage}
\renewcommand{\headrulewidth}{0.3pt}
\renewcommand{\footrulewidth}{0.0pt}
%\lhead{\footnotesize Problem Set \#3 }
\lhead{\footnotesize BUEC 311: Problem Set \#3, Shifting equilibria}
\rhead{\footnotesize \today}
\lfoot{\small \copyright }
\cfoot{}
\rfoot{\small \thepage/\pageref*{LastPage}}

\fancypagestyle{firststyle}
{
\renewcommand{\headrulewidth}{0pt}%
   \fancyhf{}
   \fancyfoot[C]{\thepage/\pageref*{LastPage}}
}

%\def\labelitemi{--}
%\usepackage{enumitem}
%\setitemize[0]{leftmargin=25pt}
%\setenumerate[0]{leftmargin=25pt}




\makeatletter
\@ifpackageloaded{hyperref}{}{%
\ifxetex
  \usepackage[setpagesize=false, % page size defined by xetex
              unicode=false, % unicode breaks when used with xetex
              xetex]{hyperref}
\else
  \usepackage[unicode=true]{hyperref}
\fi
}
\@ifpackageloaded{color}{
    \PassOptionsToPackage{usenames,dvipsnames}{color}
}{%
    \usepackage[usenames,dvipsnames]{color}
}
\makeatother
\hypersetup{breaklinks=true,
            bookmarks=true,
            pdfauthor={},
             pdfkeywords = {},
            pdftitle={Problem Set \#3: Shifting equilibria},
            colorlinks=true,
            citecolor=blue,
            urlcolor=blue,
            linkcolor=magenta,
            pdfborder={0 0 0}}
\urlstyle{same}  % don't use monospace font for urls


\setcounter{secnumdepth}{0}


%



\usepackage{setspace}

\title{\vspace{-1.5cm}\Large{BUEC 311: Business Economics, Organization
and Management}\medskip\\\Large{Problem Set \#3}
\medskip\\\Large{Shifting equilibria}
}
\date{\vspace{-.75cm}\Large{\today}}

\definecolor{light-gray}{gray}{0.8}


\begin{document}

\vspace{-5cm}\maketitle
 \tikz [remember picture,overlay]
    %\node[yshift=-1.65cm,xshift=0cm] at (current page.north)
    %\node[yshift=-1.65cm,xshift=0cm] at (current page.north)
        %or: (current page.center)
        \node[yshift=-1cm,xshift=6.5cm] at (current page.north west)
        %{\includegraphics[width=3in]{UA-ASB-COLOUR.png}};
        {\includegraphics[width=.5\paperwidth]{../images/UA-ASB-COLOUR.png}};
\vspace{-.75cm}		
		\thispagestyle{firststyle}



This week's problems are meant to give you more practice on the basic
tricks you'll need to manipulate supply and demand curves and calculate
equilibrium conditions, but these also involve shifts in equilibria. Be
methodical about your calculations and your analysis. In these
questions, I've again given you multiple steps intentionally to build
habits like checks and double-checks that you'll want to use. I won't
always give these to you on exam questions.

\begin{enumerate}
\def\labelenumi{\arabic{enumi}.}
\tightlist
\item
  Let's use the market demand curve for coffee we derived in the last
  problem set of \(Q=4000-\frac{2000p}{3}\). Let's also assume that the
  marginal cost of supplying coffee to the market is
  \(Q=\frac{2000\,p}{3}\).
\end{enumerate}

\begin{center}\includegraphics[width=468px]{week_3_problems_soln_files/figure-latex/q_1_graphs-1} \end{center}

\begin{enumerate}
\def\labelenumi{\alph{enumi})}
\tightlist
\item
  Calculate the equilibrium price and quantity (a repeat from last week)
\end{enumerate}

p=3, q=2000

\begin{enumerate}
\def\labelenumi{\alph{enumi})}
\setcounter{enumi}{1}
\tightlist
\item
  Suppose that a drought reduces the coffee crop increasing the cost of
  supplying coffee leading to a reduction in the supply function to
  \(Q=\frac{1000\,p}{3}\). Solve for the new equilibrium price and
  quantity in the market.
\end{enumerate}

p=4, q=1333.3333333 or 4000/3

\begin{enumerate}
\def\labelenumi{\alph{enumi})}
\setcounter{enumi}{2}
\tightlist
\item
  Show, on a graph, the changes in the inverse supply and demand
  functions, and label the original and new equilibrium prices and
  quantities.
\end{enumerate}

See above

\begin{enumerate}
\def\labelenumi{\alph{enumi})}
\setcounter{enumi}{3}
\tightlist
\item
  Is this a change in demand or a movement along the demand curve? Why?
\end{enumerate}

This change is a movement along the demand curve because both the new
and original equilibria lie on the same (green) demand curve.

\newpage

\begin{enumerate}
\def\labelenumi{\arabic{enumi}.}
\setcounter{enumi}{1}
\tightlist
\item
  Use the market demand curve for coffee we derived in the last problem
  set of \(Q=4000-\frac{2000p}{3}\) along with the supply function of
  \(Q=\frac{2000\,p}{3}\).

  \begin{enumerate}
  \def\labelenumii{\alph{enumii})}
  \tightlist
  \item
    Suppose a coffee tax of \$0.50 per cup is levied by the university
    in an attempt to recover some revenues lost to COVID-19. What's the
    impact on the market if you assume that the tax is charged to the
    coffee vendor based on the number of cups of coffee sold?
  \end{enumerate}

  The equilibrium price will increase to \$3.25, up from \$3, and the
  equilibrium quantity drops from 2000 to 5500/3. The revenue to
  suppliers drops from \$3 to 2.75.
\end{enumerate}

\begin{center}\includegraphics[width=468px]{week_3_problems_soln_files/figure-latex/q_2_graphs-1} \end{center}

\begin{enumerate}
\def\labelenumi{\alph{enumi})}
\setcounter{enumi}{1}
\tightlist
\item
  What happens if the same tax is added at the cash register and charged
  to patrons directly, with the proceeds of the tax remitted to the
  university?
\end{enumerate}

Nothing changes in this case from a tax applied on the vendor. Whether
it's effectively an extra charge that the consumer will pay, or an extra
input cost, we end up with the same economic outcome of p=\$3.25 and
Q=5500/3, and the before-tax revenue to suppliers is \$2.75.

\begin{center}\includegraphics[width=468px]{week_3_problems_soln_files/figure-latex/q_2_b_graphs-1} \end{center}

\begin{enumerate}
\def\labelenumi{\alph{enumi})}
\setcounter{enumi}{2}
\tightlist
\item
  Why does the equilibrium price of coffee not increase by the same
  amount as the tax?
\end{enumerate}

The tax is paid for in part by producers and in part by consumers. Only
if the demand curve were perfectly inelastic or if the supply curve was
perfectly elastic would the entire cost of the tax be reflected in the
change in price.

\newpage

\begin{enumerate}
\def\labelenumi{\arabic{enumi}.}
\setcounter{enumi}{2}
\tightlist
\item
  On slide 81 of the full supply and demand deck, I give you a graph of
  a concurrent shift in supply and demand due to a change in crude oil
  prices affecting both the input costs to produce gasoline and the
  income of consumers. Based on the remainder of the information in the
  deck:
\end{enumerate}

\begin{center}\includegraphics[width=468px]{week_3_problems_soln_files/figure-latex/q_3_graphs-1} \end{center}

\begin{enumerate}
\def\labelenumi{\alph{enumi})}
\item
  Write out the initial supply and demand functions and solve for the
  initial equilibrium price and quantity (this is a freebie, as this is
  all in the deck)

  p=93.75 Q=21.25

  \begin{enumerate}
  \def\labelenumii{\alph{enumii})}
  \setcounter{enumii}{1}
  \tightlist
  \item
    Solve for the price and quantity at the new equilibrium (not in the
    deck)
  \end{enumerate}

  p=116.25 Q=18.75

  \begin{enumerate}
  \def\labelenumii{\alph{enumii})}
  \setcounter{enumii}{2}
  \tightlist
  \item
    Now, assume the government levies a 20c/l tax (\$0.20/l in the units
    in the deck) on gasoline payable by suppliers. Calculate a new
    equlibrium price and quantity.
  \end{enumerate}
\end{enumerate}

\begin{center}\includegraphics[width=468px]{week_3_problems_soln_files/figure-latex/q_3d_graphs-1} \end{center}

Here, the question was a bit ambiguous, but let's assume that this is an
additional add-on to the double shift from above. So, since the question
specifies that consumers pay the tax, you want to do demand net of taxes
(the dotted demand curve) and then use that to calculate the supply and
then the market price is the unshifted demand at that calculated
quantity. So, p=128.75 Q=16.25.

\begin{enumerate}
\def\labelenumi{\alph{enumi})}
\setcounter{enumi}{3}
\tightlist
\item
  Has the price increased by the cost of the tax? If not, why did the
  producers not simply \textit{pass the cost on} to consumers?
\end{enumerate}

The price has increased, from p=116.25 to p=128.75, but that's less than
the value of the tax. Had the suppliers attempted to pass all of the
cost of the tax into market prices, equilibrium quantities would have
dropped more than they did, leading to lower value for suppliers.

\begin{enumerate}
\def\labelenumi{\alph{enumi})}
\setcounter{enumi}{4}
\tightlist
\item
  Now, assume the government levies a 20c/l tax (\$0.20/l in the units
  in the deck) on gasoline payable at the cash register by consumers.
  Calculate the new equlibrium price and quantity.
\end{enumerate}

\begin{center}\includegraphics[width=468px]{week_3_problems_soln_files/figure-latex/q_3c_graphs-1} \end{center}

There is no change in the equilibrium when the tax is added to supply
instead of taken away from demand. It doesn't matter whether you model
it as an additional input cost or as a consumer payment. The price has
increased, from p=116.25 to p=128.75, but that's less than the value of
the tax. Quantity has decreased from 18.75 to 16.25 as a result of the
tax.

\begin{enumerate}
\def\labelenumi{\alph{enumi})}
\setcounter{enumi}{5}
\tightlist
\item
  In the new equilibrium, what's the revenue per litre earned by
  producers and what is the price paid by consumers?
\end{enumerate}

In equilibrium, producers receive 108.75, consumers pay 128.75, and the
difference (t=20) goes to the government as tax revenue. \newpage

\begin{enumerate}
\def\labelenumi{\arabic{enumi}.}
\setcounter{enumi}{3}
\tightlist
\item
  In your own words and graphs (using the supply and demand functions
  from the class deck), describe the impact of a price ceiling on
  gasoline applied at a cost of \$0.80/l. What is likely to result from
  this price ceiling? (For now, assume the market is competitive. We'll
  abandon this in due course.)
\end{enumerate}

\begin{center}\includegraphics[width=468px]{week_3_problems_soln_files/figure-latex/q_4_graphs-1} \end{center}

There is a lot going on in this graph. The grey points indicate,
respectively, the quantities supplied and demanded at the price ceiling.
As you can see, the Quantity is far lower than the quantity consumers
would demand at \$0.80 per litre for gasoline. This will imply a
shortage and the shortage will lead to line-ups, and potentially to the
development of black markets for gasoline, since the willingness to pay
for the last unit of supply at the price ceiling is much higher than
\$0.80 per litre.




\end{document}
