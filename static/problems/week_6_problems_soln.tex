\documentclass[11pt,]{article}
\usepackage[margin=1in]{geometry}
\newcommand*{\authorfont}{\fontfamily{phv}\selectfont}
\usepackage[]{mathpazo}

\usepackage{amssymb,amsmath}
\usepackage{xcolor}

%TIKZ and Flow chart material
\usepackage{tikz}
\usetikzlibrary{shapes.geometric, arrows}
\tikzstyle{startstop} = [rectangle, rounded corners, minimum width=3cm, minimum height=1cm,text centered, draw=black, fill=red!30]
\tikzstyle{io} = [trapezium, trapezium left angle=70, trapezium right angle=110, minimum width=3cm, minimum height=1cm, text centered, draw=black,fill=white]
\tikzstyle{process} = [rectangle, minimum width=3cm, minimum height=1cm, text centered, draw=black, fill=white]
\tikzstyle{decision} = [diamond, minimum width=3cm, minimum height=1cm, text centered, draw=black, fill=white]
\tikzstyle{smalldecision} = [diamond, minimum width=1cm, minimum height=1cm, text centered, draw=black, fill=white]
\tikzstyle{arrow} = [thick,->,>=stealth]
\tikzstyle{line} = [thick,-]
\tikzstyle{dot} = [circle,inner sep=0.5pt,draw=black, fill=black]


%\tikzstyle{startstop} = [rectangle, rounded corners, minimum width=3cm, minimum height=1cm,text centered, %draw=black, fill=red!30]
%\tikzstyle{io} = [trapezium, trapezium left angle=70, trapezium right angle=110, minimum width=3cm, %minimum height=1cm, text centered, draw=black,fill=white]
%\tikzstyle{process} = [rectangle, minimum width=3cm, minimum height=1cm, text centered, draw=black, %fill=white]
%\tikzstyle{decision} = [diamond, minimum width=3cm, minimum height=1cm, text centered, draw=black, %fill=green!30]
%\tikzstyle{arrow} = [thick,->,>=stealth]




\usepackage{abstract}
\renewcommand{\abstractname}{}    % clear the title
\renewcommand{\absnamepos}{empty} % originally center
\newcommand{\blankline}{\quad\pagebreak[2]}

\providecommand{\tightlist}{%
  \setlength{\itemsep}{0pt}\setlength{\parskip}{0pt}}
\usepackage{longtable,booktabs}

\usepackage{parskip}
\usepackage{titlesec}
\titlespacing\section{0pt}{12pt plus 4pt minus 2pt}{6pt plus 2pt minus 2pt}
\titlespacing\subsection{0pt}{12pt plus 4pt minus 2pt}{6pt plus 2pt minus 2pt}

\titleformat*{\subsubsection}{\normalsize\itshape}

\usepackage{titling}
\setlength{\droptitle}{-.25cm}

%\setlength{\parindent}{0pt}
%\setlength{\parskip}{6pt plus 2pt minus 1pt}
%\setlength{\emergencystretch}{3em}  % prevent overfull lines

\usepackage[T1]{fontenc}
\usepackage[utf8]{inputenc}

\usepackage{fancyhdr}
\pagestyle{fancy}
\usepackage{lastpage}
\renewcommand{\headrulewidth}{0.3pt}
\renewcommand{\footrulewidth}{0.0pt}
%\lhead{\footnotesize Problem Set \#4 }
\lhead{\footnotesize BUEC 311: Problem Set \#4, Consumer Preferences}
\rhead{\footnotesize \today}
\lfoot{\small \copyright }
\cfoot{}
\rfoot{\small \thepage/\pageref*{LastPage}}

\fancypagestyle{firststyle}
{
\renewcommand{\headrulewidth}{0pt}%
   \fancyhf{}
   \fancyfoot[C]{\thepage/\pageref*{LastPage}}
}

%\def\labelitemi{--}
%\usepackage{enumitem}
%\setitemize[0]{leftmargin=25pt}
%\setenumerate[0]{leftmargin=25pt}




\makeatletter
\@ifpackageloaded{hyperref}{}{%
\ifxetex
  \usepackage[setpagesize=false, % page size defined by xetex
              unicode=false, % unicode breaks when used with xetex
              xetex]{hyperref}
\else
  \usepackage[unicode=true]{hyperref}
\fi
}
\@ifpackageloaded{color}{
    \PassOptionsToPackage{usenames,dvipsnames}{color}
}{%
    \usepackage[usenames,dvipsnames]{color}
}
\makeatother
\hypersetup{breaklinks=true,
            bookmarks=true,
            pdfauthor={},
             pdfkeywords = {},
            pdftitle={Problem Set \#4: Consumer Preferences},
            colorlinks=true,
            citecolor=blue,
            urlcolor=blue,
            linkcolor=magenta,
            pdfborder={0 0 0}}
\urlstyle{same}  % don't use monospace font for urls


\setcounter{secnumdepth}{0}


%



\usepackage{setspace}

\title{\vspace{-1.5cm}\Large{BUEC 311: Business Economics, Organization
and Management}\medskip\\\Large{Problem Set \#4}
\medskip\\\Large{Consumer Preferences}
}
\date{\vspace{-.75cm}\Large{\today}}

\definecolor{light-gray}{gray}{0.8}


\begin{document}

\vspace{-5cm}\maketitle
 \tikz [remember picture,overlay]
    %\node[yshift=-1.65cm,xshift=0cm] at (current page.north)
    %\node[yshift=-1.65cm,xshift=0cm] at (current page.north)
        %or: (current page.center)
        \node[yshift=-1cm,xshift=6.5cm] at (current page.north west)
        %{\includegraphics[width=3in]{UA-ASB-COLOUR.png}};
        {\includegraphics[width=.5\paperwidth]{../images/UA-ASB-COLOUR.png}};
\vspace{-.75cm}		
		\thispagestyle{firststyle}



This week's problems are more conceptual in nature, meant to ensure that
you have the definitions down first.

\begin{enumerate}
\def\labelenumi{\arabic{enumi})}
\tightlist
\item
  Consumers allocate their budgets among bundles because

  \begin{enumerate}
  \def\labelenumii{\Alph{enumii})}
  \tightlist
  \item
    more is not always better.
  \item
    bundles are the most efficient way to package goods and services.
  \item
    consumers face choices and trade-offs.
  \item
    they want to minimize trips to the store.
  \end{enumerate}
\end{enumerate}

Answer: C. This is the core of why we maximize, but will get here in
week 7.

\begin{enumerate}
\def\labelenumi{\arabic{enumi})}
\setcounter{enumi}{1}
\tightlist
\item
  The assumption of completeness means that

  \begin{enumerate}
  \def\labelenumii{\Alph{enumii})}
  \tightlist
  \item
    the consumer can rank all possible consumption bundles.
  \item
    more of a good is always better.
  \item
    the consumers can rank all affordable consumption bundles.
  \item
    all preferences conditions are met.
  \end{enumerate}
\end{enumerate}

Answer: A. The rest might be true, but are not the answer to this
question.

\begin{enumerate}
\def\labelenumi{\arabic{enumi})}
\setcounter{enumi}{2}
\tightlist
\item
  If a consumer prefers apples to bananas and prefers bananas to citrus
  fruit, in order to satisfy assumptions about preferences she has to
  prefer

  \begin{enumerate}
  \def\labelenumii{\Alph{enumii})}
  \tightlist
  \item
    bananas to apples.
  \item
    citrus fruit to bananas.
  \item
    apples to citrus fruit.
  \item
    citrus fruit to apples.
  \end{enumerate}
\end{enumerate}

Answer: C, by transitivity.

\begin{enumerate}
\def\labelenumi{\arabic{enumi})}
\setcounter{enumi}{3}
\tightlist
\item
  If a consumer weakly prefers pizza to hot dogs, and weakly prefers hot
  dogs to chicken, then he \_\_\_\_\_\_\_\_ pizza \_\_\_\_\_\_\_\_
  chicken.

  \begin{enumerate}
  \def\labelenumii{\Alph{enumii})}
  \tightlist
  \item
    likes; less than
  \item
    likes; at least as much as
  \item
    dislikes; more than
  \item
    dislikes; and is indifferent about
  \end{enumerate}
\end{enumerate}

Answer: B. Note the ``weak'' statement here.

\begin{enumerate}
\def\labelenumi{\arabic{enumi})}
\setcounter{enumi}{4}
\tightlist
\item
  The principle that ``More is better'' results in indifference curves

  \begin{enumerate}
  \def\labelenumii{\Alph{enumii})}
  \tightlist
  \item
    sloping down.
  \item
    not intersecting.
  \item
    reflecting greater preferences the further they are from the origin.
  \item
    All of the above.
  \end{enumerate}
\end{enumerate}

Answer: D \newpage
6) An indifference curve represents bundles of goods that a consumer A)
views as equally desirable. B) ranks from most preferred to least
preferred. C) prefers to any other bundle of goods. D) All of the above.

Answer: A

\begin{enumerate}
\def\labelenumi{\arabic{enumi})}
\setcounter{enumi}{6}
\tightlist
\item
  There is an indifference curve through every bundle because of the
  assumption of

  \begin{enumerate}
  \def\labelenumii{\Alph{enumii})}
  \tightlist
  \item
    transitivity.
  \item
    completeness.
  \item
    rationality.
  \item
    nonsatiation.
  \end{enumerate}
\end{enumerate}

Answer: B

\begin{enumerate}
\def\labelenumi{\arabic{enumi})}
\setcounter{enumi}{7}
\tightlist
\item
  Indifference curves are downward sloping because of the assumption of

  \begin{enumerate}
  \def\labelenumii{\Alph{enumii})}
  \tightlist
  \item
    completeness.
  \item
    transitivity.
  \item
    more is better.
  \item
    All of the above.
  \end{enumerate}
\end{enumerate}

Answer: C. Completeness tells you that I can rank all bundles, and
transitivity means they can't cross, loop, etc.

\begin{enumerate}
\def\labelenumi{\arabic{enumi})}
\setcounter{enumi}{8}
\tightlist
\item
  If two indifference curves were to intersect at a point, this would
  violate the assumption of

  \begin{enumerate}
  \def\labelenumii{\Alph{enumii})}
  \tightlist
  \item
    transitivity.
  \item
    completeness.
  \item
    Both A and B above.
  \item
    None of the above.
  \end{enumerate}
\end{enumerate}

Answer: A. See above. Completeness isn't relevant here.

\begin{enumerate}
\def\labelenumi{\arabic{enumi})}
\setcounter{enumi}{9}
\tightlist
\item
  Indifference curves close to the origin are \_\_\_\_\_\_\_\_ those
  farther from the origin because of \_\_\_\_\_\_\_\_.
\end{enumerate}

\begin{enumerate}
\def\labelenumi{\Alph{enumi})}
\tightlist
\item
  better than; transitivity
\item
  worse than; nonsatiation
\item
  better than; completeness
\item
  worse than; transitivity
\end{enumerate}

Answer: B. More is better, so up and to the right are better bundles.

\begin{enumerate}
\def\labelenumi{\arabic{enumi})}
\setcounter{enumi}{10}
\tightlist
\item
  The indifference curves for left shoes and right shoes would most
  likely be
\end{enumerate}

\begin{enumerate}
\def\labelenumi{\Alph{enumi})}
\tightlist
\item
  upward sloping and concave to the origin.
\item
  downward sloping and convex to the origin.
\item
  downward sloping and straight lines.
\item
  L-shaped.
\end{enumerate}

Answer: D. These are perfect complements (assuming you don't have two
left feet.) \newpage

\begin{enumerate}
\def\labelenumi{\arabic{enumi})}
\setcounter{enumi}{11}
\tightlist
\item
  If two bundles are on the same indifference curve, then
\end{enumerate}

\begin{enumerate}
\def\labelenumi{\Alph{enumi})}
\tightlist
\item
  the consumer derives the same level of utility from each.
\item
  the consumer derives the same level of ordinal utility from each but
  not the same level of cardinal utility.
\item
  no comparison can be made between the two bundles since utility cannot
  really be measured.
\item
  the MRS between the two bundles equals one.
\end{enumerate}

Answer: A.

\begin{enumerate}
\def\labelenumi{\arabic{enumi})}
\setcounter{enumi}{12}
\tightlist
\item
  Adrian's utilities of two consumption bundles are 50 and 100
  respectively. This implies that
\end{enumerate}

\begin{enumerate}
\def\labelenumi{\Alph{enumi})}
\tightlist
\item
  Adrian prefers the first bundle.
\item
  Adrian prefers the second bundle.
\item
  Adrian likes the second bundle twice as much.
\item
  Adrian likes the first bundle twice as much.
\end{enumerate}

Answer: B. Higher utility is better, but the ranking might not be
cardinal.

\begin{enumerate}
\def\labelenumi{\arabic{enumi})}
\setcounter{enumi}{13}
\tightlist
\item
  Joe's income is \$500, the price of food (F, y-axis) is \$2 per unit,
  and the price of shelter (S, x-axis) is \$100. Which of the following
  represents his budget constraint?
\end{enumerate}

\begin{enumerate}
\def\labelenumi{\Alph{enumi})}
\tightlist
\item
  500 = 2F + 100S
\item
  F = 250 - 50S
\item
  S = 5 - .02F
\item
  All of the above.
\end{enumerate}

Answer: D. They are just transformations of the same relationship.

\begin{enumerate}
\def\labelenumi{\arabic{enumi})}
\setcounter{enumi}{14}
\tightlist
\item
  Joe's income is \$500, the price of food (F, y-axis) is \$2 per unit,
  and the price of shelter (S, x-axis) is \$100. Which of the following
  represents his marginal rate of transformation of food for shelter?
\end{enumerate}

\begin{enumerate}
\def\labelenumi{\Alph{enumi})}
\tightlist
\item
  -5
\item
  -50
\item
  -.02
\item
  None of the above.
\end{enumerate}

Answer: B. -100/2. Income doesn't matter for calculating the MRT.

\begin{enumerate}
\def\labelenumi{\arabic{enumi})}
\setcounter{enumi}{15}
\tightlist
\item
  With respect to consuming food and shelter, two consumers face the
  same prices and both claim to be in equilibrium. We therefore know
  that
\end{enumerate}

\begin{enumerate}
\def\labelenumi{\Alph{enumi})}
\tightlist
\item
  they both have the same marginal utility for food.
\item
  they both have the same marginal utility for shelter.
\item
  they both have the same MRS of food for shelter.
\item
  All of the above.
\end{enumerate}

Answer: C. If pprices are the same, then MRT is the same for both
consumers. Since they are at the optimal point, MRS=MRT for both, so it
must be the case that MRS is the same both of them. Since utility isn't
cardinal, it need not be the case that they have the same marginal
utilities for either individual good.

\begin{enumerate}
\def\labelenumi{\arabic{enumi})}
\setcounter{enumi}{16}
\tightlist
\item
  An individual's demand curve for a good can be derived by measuring
  the quantities selected as
\end{enumerate}

\begin{enumerate}
\def\labelenumi{\Alph{enumi})}
\tightlist
\item
  the price of the good changes.
\item
  the prices of substitute goods change.
\item
  income changes.
\item
  All of the above.
\end{enumerate}

Answer: A. You'd calculate shifts in the demand curve using B and C.

\begin{enumerate}
\def\labelenumi{\arabic{enumi})}
\setcounter{enumi}{17}
\tightlist
\item
  Behavioral economics extends traditional economic models by
\end{enumerate}

\begin{enumerate}
\def\labelenumi{\Alph{enumi})}
\tightlist
\item
  including insights from psychology and human cognition models.
\item
  modeling behavior rather than prices.
\item
  admitting that individuals are irrational.
\item
  admitting that incentives are very important.
\end{enumerate}

Answer: A.

\begin{enumerate}
\def\labelenumi{\arabic{enumi})}
\setcounter{enumi}{18}
\tightlist
\item
  In Spain, people are considered organ donors unless they explicitly
  indicate they do not want to be. In the United States, people are only
  considered organ donors if they explicitly indicate they wish to be.
  Behavioral economics would suggest that everything else equal,
\end{enumerate}

\begin{enumerate}
\def\labelenumi{\Alph{enumi})}
\tightlist
\item
  the opt-in system of Spain would generate more organ donors. as a
  percentage of the adult population.
\item
  the opt-in system of the United States would generate more organ
  donors as a percentage of the adult population.
\item
  the opt-out system of Spain would generate more organ donors. as a
  percentage of the adult population.
\item
  the opt-out system of the United States would generate more organ
  donors as a percentage of the adult population.
\end{enumerate}

Answer: C. People tend to be less likely to opt-out.

\begin{enumerate}
\def\labelenumi{\arabic{enumi})}
\setcounter{enumi}{19}
\tightlist
\item
  In behavioral economics, salience is best exemplified by
\end{enumerate}

\begin{enumerate}
\def\labelenumi{\Alph{enumi})}
\tightlist
\item
  consumers responding differently when posted prices increase rather
  than when prices increase because of sales tax increases.
\item
  consumers responding the same regardless of how prices change.
\item
  the end of a controlled experiment.
\item
  consumers responding differently when income increases permanently
  rather than temporarily.
\end{enumerate}

Answer: A




\end{document}
